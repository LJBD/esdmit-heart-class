\section*{Dodatek B: Spis zawartości dołączonego nośnika CD.}
\addcontentsline{toc}{section}{Dodatek B: Spis zawartości dołączonego nośnika CD.}
\label{dodatekB}

\qquad Nośnik CD, traktowany jako suplement do projektu, zawiera następującą strukturę katalogów, będącą bezpośrednim zapisem repozytorium projektowego: 

\begin{enumerate}
	\item Julia - w środku znajduje się kompletna implementacja w języku Julia. W osobnym folderze zostały umieszczone pliki związane z algorytmem SVM.
	\item Literatura - zawiera część plików wymienionych w Bibliografii niniejszego raportu.
	\item Matlab - ten katalog mieści 3 kategorie plików:
	\begin{enumerate}
		\item skrypty wykorzystywane przez ten projekt,
		\item skrypty pomocnicze, używane do testowania lub tworzenia wykresów,
		\item bibliotekę LIBSVM w postaci skompilowanych obiektów w środowisku Windows.
	\end{enumerate}
	\item Python - podobnie jak w katalogu Julia, znajdują się tutaj pliki związane z implementacją w Pythonie. Osobne foldery zawierają implementację algorytmów G-średnich i SVM. Wszystkie katalogi posiadają również pliki \texttt{\_\_init\_\_.py} potrzebne, aby Python rozpoznawał je jako lokalizacje swoich plików źródłowych.
	\item Raport - w środku znajdują się pliki źródłowe niniejszego raportu w języku LaTeX, wszystkie użyte tu grafiki oraz sam raport w formacie PDF. Znajdują się tam również pliki z różnicami, które zostały wprowadzone w projekcie ubiegłorocznym.
	\item ReferencyjneDane - ten katalog zawiera wszystkie paczki danych, które zostały zapisane w czasie działania projektu ubiegłorocznego w języku C++.
	\item SVM\_models - znajdują się tam modele stosowane przez algorytm SVM, zapisane w formacie tekstowym.
	\item Pliki w głównym katalogu: README.md oraz .gitignore.
\end{enumerate}